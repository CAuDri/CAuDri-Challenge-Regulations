\chapter{Competition Schedule}

In this chapter, the general schedule execution of the competition is
described.

\section{Training}

In order to guarantee safe and fair training conditions, the training sessions
are divided into time slots. The number of teams allowed on the track at the
same time and the length of the slots will be announced on the website before
the competition. The commission might change the slots and the number of teams
on the track without further notice. In case of clear violations of training
slots, the commission may issue penalties which will be subtracted from the
final score of the respective teams. In case of repetitive violations of slots
or if team members endanger other teams or their equipment, the commission may
expel single team members or whole teams from the competition.

\section{Qualifying}

\colorbox{yellow}{There will be no qualifying in \the\year{}.}

\section{Competition}
\subsection{Preparations}

15 min before the beginning of the competition, the teams must hand in their vehicles at the “parc fermé”. No modifications of the vehicles must be made after this point. Batteries must be separated from the system, the vehicle must be switched off. All external tools must be removed from the vehicle, all wireless communication on board the vehicles (Wi-Fi, Bluetooth, etc.) must be switched off or removed, except for the remote control communication. The remote control must be placed next to the vehicle in switched off state. When handing in the vehicle, the teams must make a definite statement to the head referee in which events they would like to participate. This is to ensure a smooth execution of the competition.

\subsection{Start Scheduling System}
\label{start_scheduling}

A traffic-light-like start scheduling system will signal the teams when to pick
up their vehicle at the “parc fermé” and begin to prepare for starting. The
traffic light will show the following stages:

\begin{itemize}
	\item 1. Red: No preparation necessary

	\item 2. Yellow: The vehicle must be prepared for the start. The team picks up their vehicle at the “parc fermé”. Time budget for preparation is 5 min. The teams may change to fully charged batteries in this context. However, no additional preparation (using external tools) is allowed at this stage. The idle, but ready, vehicle must be brought to the start box at this point. Timing will start, regardless whether the vehicle is ready or not.\\
	      \colorbox{yellow}{\parbox{0.8\colorboxwidth}{
			      If an active wifi connection or other external tool is needed to start the vehicle, a penalty will be applied equal to that of an active wifi connection during the competition.}}

	\item 3. Green: When showing “green” the gate of the start box might open at any time. After each event, the vehicle must be returned to the “parc fermé” immediately. Batteries must again be separated from the system, the vehicle must be switched off. The remote control must be placed next to the vehicle in switched off state.
\end{itemize}

\subsection{Start Box}
\label{start_box}

The start box is separated from the track by physical barriers. Up to two team
members are allowed to prepare the start of the vehicle in the start box. To
the front of the start box is an openable gate, marked with a traffic sign and
a matrix barcode (cf. Section \ref{fig_start_box_markings}). An attempt starts
with the opening of the gate. The start box exit can be separated from the
track by a solid white line. This line may be crossed to enter the track. The
gate of the start box remains open for 30 s. An attempt is canceled if:

\begin{itemize}
	\item The vehicle has not been placed ready to start in the box when the gate opens,
	\item The gate is forced open by a vehicle
	\item The vehicle fails to leave the box while the gate is open
	\item RC-mode is activated inside the start box
\end{itemize}

Penalties will not be applied until the vehicle passes the start line (i.e.
collisions in the start box or driving outside of the right lane before passing
the start line do not reduce the overall score).

%\subsection{Order of Events}
%
%
%As described, “Carolo-Master-Cup” and “Carolo-Basic-Cup” do not share the same track. Thus, the teams of the “Carolo-Basic-Cup” will perform their dynamic events on parallel tracks initially. In case of a large number of participating teams, the dynamic events of the “Carolo-Basic-Cup” may be performed individually, prior to the official evening event of the “Carolo-Cup”. Subsequently, the competition area will be converted to a large circuit for the dynamic events of the “Carolo-Master-Cup”. The first event of both competitions is the “Free Drive and Parking”: One team after another starts its event according to the regulations in Section 5.2. The order of teams within each competition is fixed. The calls for preparation and start are made according to the start scheduling system, described above. In both competitions, the event “Obstacle Evasion Course” will follow next. The starting order will stay the same as in “Free Drive and Parking”. The first attempt can be canceled according to the regulations in Sections 5.2.2.2 and 5.3.7.2. The respective team is moved to the end of the schedule and will be called again to attempt a second run. 
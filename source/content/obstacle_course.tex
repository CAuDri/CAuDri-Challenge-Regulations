\newpage

\section{Obstacle Evasion Course}

The event “Obstacle Evasion Course” extends the track of the Free Drive event
with additional elements which need to be considered during the driving task.
Parking maneuvers shall not be performed within this event. Static and dynamic
obstacles are added to the rural road scenario. The track does additionally
contain at least one suburban section at this point. All definitions concerning
the course of the road maintain validity. There will be at least 1000 mm track
length between obstacles. The additional elements are spaced at least 1000 mm
apart as well and do not overlap. Oncoming traffic is not to be expected,
except when passing barred areas inside the suburban scenario.

\subsubsection{Suburban Scenario}

The suburban area is a special section of the track, containing additional
elements compared to the track design of the rural road scenario. Beginning and
end of suburban areas are defined by markings on the road surface (cf. Section
\ref{fig_road_markings}) and according traffic signs (cf. Section
\ref{traffic_signs}).

The speed limit within the suburban section, as indicated by the traffic signs
has to be scaled by 1:10 (i.e. a speed limit of 30 km/h corresponds to 0.83
m/s). In addition to the speed limits depicted in the signs marking the
suburban scenario, other numeric signs in steps of 10 km/h might appear (e.g. a
speed limit of 20 km/h). Speed limit zones begin and end at the road markings,
as depicted in Section \ref{fig_speed_limit_zone}. The according traffic signs
will be placed at those positions. Elements of the suburban scenario will not
be located on uphill and downhill grades.

\subsection{Elements of the Obstacle Evasion Course}
\label{elements_obstacle_evasion}
\subsubsection{Static Obstacles}

During this event, a number of static obstacles will be placed in the right
lane, in the left lane and outside of the track. The body of each obstacle
consists of white cardboard with dimensions as specified in the appendix
(Section \ref{fig_obstacle_dimensions}). Obstacles can be fixed on the ground.
The obstacles are not always placed exactly in a specific lane, however under
no circumstance can both lanes be blocked. In this sense, static obstacles
outside the track are no artifacts in the sense of Section \ref{artifacts}.
Thus, the described minimum distance to lane markings for artifacts does not
apply.

Obstacles may force the vehicle to change lanes. Lane changes must be indicated
using the turn indicators. Passing maneuvers must be executed without touching
an obstacle. They must be completed after a maximum distance of 2 m after
having passed the obstacle.

\subsubsection{Dynamic Obstacles}

Apart from static obstacles, at least one dynamic obstacle is present on the
track. Its shape resembles the static obstacles (“driving white cardboard box”)
and it can be encountered in both lanes and in combination with other track
elements, as long as this is not explicitly excluded. It moves at a speed of
0.6 m/s. Dynamic obstacles do not execute lane changes and do not perform any
passing maneuver. Dynamic obstacles can stop temporarily and potentially block
the right lane. It may be passed, but not in intersections. Passing maneuvers
in intersections are penalized. A dynamic obstacle will not block both lanes in
combination with a static obstacle, unless passing is prohibited in the area
(cf. Section \ref{no_passing_zones}). Thus, allowed passing maneuvers can
always be executed without encountering an obstacle on the left lane. The
passing maneuver is subject to the same regulations as when passing a static
obstacle.

\subsubsection{Intersections of the Rural Road Scenario}
\label{intersection_rural}

Sections of the track can be part of intersections with other parts of the
track. The respective lanes meet at angles between 70° and 90°.

An intersection possesses three to four entries or exits respectively. Design
and layout of the intersections of the rural road scenario are shown in the
appendix (Section \ref{intersection_rural}). Left and right lane boundaries of
intersecting lanes can be connected through a rounded transition with a radius
of about 100 mm. Intersections of the rural road scenario must be crossed
driving straight. Entries to intersections can display stop lines. These lines
are 36 mm to 40 mm wide and cross one lane completely.

Additionally, a stop line is complemented by a traffic sign (stop sign, cf.
Section \ref{fig_traffic_signs}). Entries without a stop line are not marked
separately. The right of way is only announced by the respective traffic sign.
If a stop line is located in the own lane, the vehicle must stop for at least 3
s. The front of the vehicle must be located in front of the stop line, however
the distance must not be greater than 150 mm.

The right of way of a dynamic obstacle must be respected at an intersection, if
the dynamic obstacle is located within the defined area (cf. Section
\ref{fig_intersection_give_way}). If the vehicle does not possess the right of
way, it must wait until the dynamic obstacle has completely crossed the
intersection. Only one dynamic obstacle at a time can be present at an
intersection.

\subsubsection{No-Passing Zones}
\label{no_passing_zones}

Sections of the track, not only in the suburban area, can be defined as
no-passing zones. Corresponding traffic signs and lane markings will indicate
such sections (cf. Section \ref{lane_markings}). In sections with a solid
center line (a solid line within a double center line facing the ego lane)
obstacles must not be passed. However, if a passing maneuver has been started
before a no-passing zone, the vehicle is allowed to return to the right lane in
any case.

In a no-passing zone, the dynamic obstacle must be followed at a distance of at
least 300 mm until the end of the zone. Static obstacles will not block the
right lane in no-passing zones. Since passing is prohibited, a combination of a
dynamic obstacle in the right lane and a static object in the left lane can
occur, temporarily blocking the whole track.

\subsubsection{Two-lane Expressway}

Sections of the rural scenario, can be defined as an expressway. The beginning
and end of such sections will be indicated by traffic signs (cf. Section
\ref{fig_traffic_signs}). Expressways are a planar and mostly straight section
of at least 10 m length, without any sharp turns. Any distinctive curve will be
supported with traffic signs, as described in Section \ref{traffic_signs}.
Vehicles on the expressway have right of way, no stop lines will be
encountered. No obstacles will be present in the right lane of this section.
Since the track is assumed to be a two-lane expressway, the vehicles must stay
in the right lane all the time.

\subsection{Additional Elements of the Suburban Scenario}
\label{elements_suburban_scenario}
\subsubsection{Traffic Signs}

In addition to the traffic signs defined in Section \ref{traffic_signs} and
\ref{intersection_rural}, the suburban scenario contains several other traffic
signs which must be respected. Each traffic sign defines the beginning of the
connected elements as defined in the following sections. Traffic signs can only
occur in combination with their connected element. The exact dimensions and
positioning are defined in the appendix of this document (cf. Section
\ref{fig_traffic_signs}). Distances for longitudinal distances are measured on
the right hand lane marking. Each Traffic Sign of the suburban scenario is
complemented with specific markings on the road surface. Those markings must
not have the same distance to the corresponding element as the traffic sign
(cf. Section \ref{turning}). See the following sections for the according
specifications.

\subsubsection{Barred Area}

In addition to obstacles, the suburban scenario can contain barred areas on
straight sections of the track. These areas block one lane for a length of max.
2000 mm, measured along the outer lane marking. The areas must be passed just
as a regular obstacle. Barred areas are marked with a 18 mm to 20 mm wide
trapezoidal outline, filled with 36 mm to 40 mm wide white markings with black
spacing. For shape and dimensions see Section \ref{fig_barred_area}. The areas
are at least 150 mm wide and are always connected with the left or right lane
boundaries. Oncoming traffic has the right of way at barred areas, indicated by
a corresponding traffic sign (cf. Section \ref{fig_traffic_signs}).

If a dynamic obstacle is located within 1000 mm of the beginning of the barred
area, the vehicle has to wait. Switching lanes is only allowed with an empty
left lane, oncoming traffic must have completely passed. The desired passing
maneuver has to be indicated while waiting by flashing the left turn
indicators. Only one dynamic obstacle at a time can occur at a barred area. If
the vehicle is able to pass a barred area without leaving the own lane or
driving over the markings, the vehicle may continue along the barred area even
in case of oncoming traffic.

\subsubsection{Crosswalk}

In a suburban area, one or more crosswalks may be present. These are marked
with several 36 mm to 40 mm wide and 400 mm long white markings parallel to the
direction of travel which are spaced 40 mm apart (cf. Section
\ref{fig_crosswalk}). A crosswalk is indicated by a corresponding traffic sign
(cf. Section \ref{fig_traffic_signs}). On the roadside at each crosswalk
“pedestrians” may wait to cross the road. For this purpose two areas are
defined which may contain relevant pedestrians.

A “pedestrian” is depicted by a small white cardboard box in analogy to the
static obstacles. In addition, each pedestrian is marked with a pictograph, in
order to facilitate its detection (cf. Section \ref{fig_pedestrians}). Multiple
pedestrians can be located on the right- as well as on the left-hand side of
the crosswalk. Pedestrians will always be clearly distinguishable from the view
of the approaching vehicle. Only if at least one pedestrian is present in the
defined zones, the vehicle must stop in front of the crosswalk. Stopping must
be performed with the same regulations as at intersections. Pedestrians start
crossing only after the vehicle has stopped. If all relevant pedestrians have
crossed in front of the vehicle, the vehicle may continue. Driving on before
all pedestrians start to cross and have cleared the crosswalk will be
penalized.

\subsubsection{Extended Regulations at Intersections}

In addition to the requirements arising from stop lines, there can be different
regulations for the right of way at intersections in the suburban scenario.
Three types of intersections have to be considered:

\begin{itemize}
	\item Intersections with stop lines (cf. Section \ref{intersection_rural})
	\item Intersections with priority road and give-way lines
	\item Intersections without regulations by road markings or signs (priority to the
	      right)
\end{itemize}

Dimensions and layout of the additional intersections are displayed in the
appendix (cf. Section \ref{additional_intersections}). Stop lines and give-way
lines at priority roads are also announced by traffic signs (cf. Sections
\ref{fig_traffic_signs}). A give-way line is 36 mm to 40 mm wide and consists
of 80 mm long dashes, interrupted by 60 mm long gaps. Stop and give-way lines
occur in pairs at opposing intersection entries, unless the priority road
displays a mandatory direction and requires turning (cf. next Section). At a
give-way line, the vehicle must stop for at least 1 s.

Dynamic obstacles must be considered in any type of intersection. If an
intersection does not contain any indication of priority by road markings or
signs, priority to the right is to be applied. There will be no traffic signs
to announce such intersections, while all four arms of the intersection will
display a give-way line. The requirement to stop and potentially give the right
of way to dynamic obstacles must still be respected. Scenarios which yield
ambiguous regulations of the right of way will not be encountered.

\subsubsection{Turning}
\label{turning}

In addition to the intersections described above, intersections in the suburban
scenario can have a mandatory direction to cross the intersection. Different
scenarios are shown in Section \ref{fig_intersection_mandatory}. This will be
announced by a corresponding traffic sign and a marking on the road surface
(cf. Sections \ref{traffic_signs} and \ref{fig_road_markings}). Vehicles will
have to turn left or right according to these regulations. In the intersection,
the mandatory direction will additionally be indicated by dashed turn lines
that continue the center line and the right lane boundary. Turn lines cannot be
missing.

\subsubsection{Speed Control}

Within a suburban area, the vehicle has to adhere to the given speed limit.
Devices for controlling the speed of the vehicle might be present.

\subsection{Execution of the Event}

\subsubsection{Start}

The starting order of the teams will be announced by the commission, visualized
using the start scheduling system (cf. Section \ref{start_scheduling}) during
the competition. The vehicle must be placed in the start box, located next to
the track (cf. Section \ref{start_box}). The attempt is started by a judge or a
referee, signaled with opening of the gate of the start box.\\ It is not
strictly necessary to detect the presence of the markings on the gate, the
vehicle only has to detect when the gate is opened.(cf. Section
\ref{fig_start_box_markings})

\subsubsection{Attempts}

The attempt may be canceled while the gate of the start box is open. The team
is then allowed a second attempt, after all other teams have completed their
first attempt.\\\colorbox{yellow}{\parbox{\colorboxwidth}{The commission may
		arbitrarily choose the time of the second attempt if this is required to comply
		with the schedule. The affected team will be given at least 10 minutes to
		prepare for the second attempt after being informed by the commission. The
		vehicle does not have to be placed back at the "parc-fermé" and may be modified
		by the team while preparing for the second attempt.}} Canceling an attempt is
penalized (cf. Section \ref{obstacle_scoring}). A missed start results in a
second attempt automatically.

\subsection{RC-Mode}

In case the vehicle is not able to continue following the track on its own, the
team may activate RC-mode in order to get the vehicle back into normal
behavior. If the vehicle does not return into the right driving lane on its
own, RC-mode must be activated immediately. Distances travelled outside of the
driving lane will otherwise be subtracted from the total distance covered.
Bonuses cannot be earned by vehicle behavior shown in RC-mode. Additionally,
skipping challenges of the obstacle course (by not driving in the right lane)
will be punished with the penalty designated for the respective element. Each
activation of RC-mode is penalized. RC-mode is subject to the regulations in
Section \ref{rc_mode}.

\subsection{Scoring}{
	\label{obstacle_scoring}

	\renewcommand*\footnoterule{} % No line above footnotes
	\newcommand{\topstrut}{\rule{0pt}{3.5ex}}
	\rowcolors{0}{white}{lightgray!40} % Alternate row colors

	Each team will start the event with a fixed number of base points. The base
	points are predetermined by the commission and will depend on the length of the
	track and the number of occuring elements. The total distance covered by the
	vehicle will have no influence on the scoring.

	\subsubsection{Timing}
	Each team will be given a \textbf{5-minute time limit} to complete
	\textbf{three laps} around the track. After the vehicle has completed all three
	laps, the attempt is over and no further points will be awarded. Timing of the
	event starts with the opening of the gate of the start box, described in
	section \ref{start_box}.

	\subsubsection{Evaluation}
	When the vehicle passes one of the elements described in section
	\ref{elements_obstacle_evasion} and \ref{elements_suburban_scenario}, it will
	receive either a positive, neutral or negative evaluation.

	Any time the vehicle complies with all the requirements of a certain element,
	it will receive a positive evaluation and gain points.\\ If the vehicle fails
	to comply with any of the requirements, it will generally receive a neutral
	evaluation and no points will be awarded.\\ In some cases, the vehicle may
	receive a negative evaluation and points will be deducted.

	The following section \ref{obstacle_scoring_guidelines} provides an overview of
	the scoring guidelines for each element. The bottom row of each table shows the
	amount of points that will be awarded or deducted for each evaluation.

	The final score of each team will be the sum of the base points and the points
	gained or lost during the event. Only after a vehicle has completed at least
	one full lap will the attempt be valid and any points awarded. In case the
	vehicle has left the track or skipped certain parts of it, the commission will
	decide whether the attempt was valid.

	\subsection{Scoring Guidelines}
	\label{obstacle_scoring_guidelines}

	\subsubsection*{Static Obstacles}
	\begin{table}[H]
		\begin{tabularx}{\textwidth}{XXX}
			\toprule
			\textbf{Positive}                & \textbf{Neutral}             & \textbf{Negative}          \\
			\midrule
			Use of turn indicators           & Wrong use of turn indicators & Collision with an Obstacle \\
			Successfully passed the Obstacle & Merging distance > 2m        &                            \\
			                                 &                              &                            \\
			\topstrut
			\textbf{+10}                     & \textbf{0}                   & \textbf{-10}               \\
			\bottomrule
		\end{tabularx}
	\end{table}

	\subsubsection*{Dynamic Obstacles}
	\begin{table}[H]
		\begin{tabularx}{\textwidth}{XXX}
			\toprule
			\textbf{Positive}                & \textbf{Neutral}                & \textbf{Negative}          \\
			\midrule
			Use of turn indicators           & Wrong use of turn indicators    & Collision with an Obstacle \\
			Successfully passed the Obstacle & Merging distance > 2m           &                            \\
			                                 & Obstacle passed in intersection &                            \\
			\topstrut
			\textbf{+15}                     & \textbf{0}                      & \textbf{-10}               \\
			\bottomrule
		\end{tabularx}
	\end{table}

	\subsubsection*{Intersections of the Rural Road Scenario}
	\begin{table}[H]
		\begin{tabularx}{\textwidth}{XXX}
			\toprule
			\textbf{Positive}                                           & \textbf{Neutral}                                                   & \textbf{Negative}                                              \\
			\midrule
			Vehicle went straight through the intersection              & Vehicle made a wrong turn                                          & Collision with an Obstacle                                     \\
			\textit{Vehicle stopped at the stop line}\footnotemark[1]   & \textit{Vehicle stopped for < 3s}\footnotemark[1]                  & \textit{Vehicle did not stop at the stop line}\footnotemark[1] \\
			                                                            & \textit{Distance from stop-line > 15cm}\footnotemark[1]            &                                                                \\
			\textit{Vehicle respected the right-of-way}\footnotemark[2] & \textit{Vehicle did not respect the right-of-way}\footnotemark[2]  &                                                                \\
			                                                            & \textit{Obstacle has not cleared the intersection}\footnotemark[2] &                                                                \\
			\topstrut
			\textbf{+10}                                                & \textbf{0}                                                         & \textbf{-10}                                                   \\
			\bottomrule
		\end{tabularx}
	\end{table}

	\footnotetext[1]{In case a stop line is present}
	\footnotetext[2]{In case a dynamic Obstacle is present}

	\subsubsection*{Extended Intersections}
	\begin{table}[H]
		\begin{tabularx}{\textwidth}{XXX}
			\toprule
			\textbf{Positive}                                           & \textbf{Neutral}                                                   & \textbf{Negative}                                      \\
			\midrule
			Vehicle stopped at the stop or give-way line                & \textit{Distance from stop-line > 15cm}\footnotemark[1]            & \textit{Did not stop at the stop line}\footnotemark[1] \\
			                                                            & \textit{Stopped for < 3s at stop line}\footnotemark[1]             &                                                        \\
			                                                            & \textit{Stopped for < 1s at give-way line}\footnotemark[3]         &                                                        \\
			\textit{Vehicle respected the right-of-way}\footnotemark[2] & \textit{Vehicle did not respect the right-of-way}\footnotemark[2]  & Collision with an Obstacle                             \\
			                                                            & \textit{Obstacle has not cleared the intersection}\footnotemark[2] &                                                        \\
			\textit{Vehicle took the right turn}\footnotemark[4]        & \textit{Vehicle took the wrong turn}\footnotemark[4]               &                                                        \\
			                                                            &                                                                    &                                                        \\
			\topstrut
			\textbf{+20}                                                & \textbf{0}                                                         & \textbf{-10}                                           \\
			\bottomrule
		\end{tabularx}
	\end{table}

	\footnotetext[1]{In case a stop line is present}
	\footnotetext[2]{In case a dynamic Obstacle is present}
	\footnotetext[3]{In case a give-way line is present}
	\footnotetext[4]{In case the intersection has a mandatory turn}

	\subsubsection*{No-Passing Zones}
	\begin{table}[H]
		\begin{tabularx}{\textwidth}{XXX}
			\toprule
			\textbf{Positive}                & \textbf{Neutral}                   & \textbf{Negative}          \\
			\midrule
			Vehicle stayed in the right lane & Vehicle started a passing maneuver & Collision with an Obstacle \\
			Distance to Obstacle > 30cm      & Distance to Obstacle < 30cm        &                            \\
			                                 &                                    &                            \\
			\topstrut
			\textbf{+5}                      & \textbf{0}                         & \textbf{-10}               \\
			\bottomrule
		\end{tabularx}
	\end{table}

	\clearpage

	\subsubsection*{Barred Area}
	\begin{table}[H]
		\begin{tabularx}{\textwidth}{XXX}
			\toprule
			\textbf{Positive}                                           & \textbf{Neutral}                                                  & \textbf{Negative}          \\
			\midrule
			Vehicle did not enter the barred area                       & Vehicle entered the barred area                                   & Collision with an Obstacle \\
			\textit{Vehicle respected the right-of-way}\footnotemark[1] & \textit{Vehicle did not respect the right-of-way}\footnotemark[1] &                            \\
			Use of turn indicators                                      & Wrong use of turn indicators                                      &                            \\
			\topstrut
			\textbf{+15}                                                & \textbf{0}                                                        & \textbf{-10}               \\
			\bottomrule
		\end{tabularx}
	\end{table}

	\subsubsection*{Crosswalk}
	\begin{table}[H]
		\begin{tabularx}{\textwidth}{XXX}
			\toprule
			\textbf{Positive}                                               & \textbf{Neutral}                                                   & \textbf{Negative}           \\
			\midrule
			\textit{Vehicle stopped at the crosswalk}\footnotemark[2]       & \textit{Vehicle stopped for < 3s at the crosswalk}\footnotemark[2] & Collision with a pedestrian \\
			\textit{Pedestrians have cleared the crosswalk}\footnotemark[2] & \textit{Stopping distance from crosswalk > 15cm}\footnotemark[2]   &                             \\
			                                                                & \textit{Pedestrians have not fully crossed}\footnotemark[2]        &                             \\
			\topstrut
			\textbf{+15}                                                    & \textbf{0}                                                         & \textbf{-10}                \\
			\bottomrule
		\end{tabularx}
	\end{table}

	\footnotetext[1]{In case a dynamic Obstacle is present}
	\footnotetext[2]{In case a pedestrian is present}

	\subsection*{Additional Penalties}
	\begin{table}[H]
		\begin{tabular}{@{}lccc@{}}
			\toprule
			\textbf{Violation}            & \textbf{Maximum Count} &  & \textbf{Penalty}   \\
			\midrule
			Second Attempt                & 1                      &  & -0.5 x base points \\
			Active WiFi Connection        & 1                      &  & -0.5 x base points \\
			Exceeding Speed-Limit         & $\infty$               &  & -10                \\
			Activation of RC-Mode         & $\infty$               &  & -5                 \\
			Leaving the right lane        & 10                     &  & -5                 \\
			Collision with road sign      & $\infty$               &  & -5                 \\
			Falsely using turn indicators & 10                     &  & -5                 \\
			\bottomrule
		\end{tabular}
	\end{table}
	\clearpage
}